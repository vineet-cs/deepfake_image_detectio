\section{Motivation}
\vspace{1.5 cm}
The rise of deepfake technology poses a significant threat to various aspects of society, including politics, entertainment, and personal privacy. Deepfake algorithms can generate highly realistic fake images and videos, making it increasingly difficult to distinguish between authentic and manipulated media. This capability raises concerns about the potential misuse of deepfakes for spreading misinformation, manipulating public opinion, and defaming individuals.
\vspace{1.0cm}

In light of these challenges, the development of robust deepfake detection methods is essential to safeguard the integrity of digital media and mitigate the harmful effects of misinformation. Detecting deepfake images and videos requires advanced algorithms capable of identifying subtle inconsistencies and artifacts introduced during the manipulation process. By accurately detecting deepfakes, we can empower individuals, organizations, and platforms to authenticate media content, preserve trust, and combat the spread of disinformation.
\vspace{1.0cm}

Our project aims to contribute to this critical area of research by exploring novel techniques for deepfake image detection.

\vspace{1.0cm}
Through rigorous experimentation and analysis, we seek to enhance our understanding of deepfake generation processes and improve the robustness of detection methods against evolving manipulation techniques.