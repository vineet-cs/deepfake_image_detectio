\begin{center}
    \Large\textbf{Abstract}
    
\end{center}

\vspace{1cm}

\large

\vspace{1.5 cm}
In recent years, the rise of AI-driven face manipulation techniques, such as
\textbf{DeepFakes}, has posed serious threats to society, emphasizing the need
for effective detection methods. In this paper, we propose a novel
approach called \textbf{Bi-granularity artifacts (BiG-Arts)} for detecting DeepFake
videos. Our method exploits a key observation: DeepFake generation often
leaves bi-granularity artifacts, comprising intrinsic and extrinsic
components.
\vspace{1.0cm}
The intrinsic artifacts stem from common model operations like up-
convolution, while extrinsic artifacts arise from post-processing steps
blending the synthesized face into the original video. To address this, we
frame DeepFake detection as a \textbf{multi-task learning problem}, aiming to
predict both artifact types simultaneously.
\vspace{1.0cm}
By leveraging the detection of Bi-granularity artifacts, our method
demonstrates significant improvements in both within-dataset and cross-
dataset scenarios. Extensive experiments on various DeepFake datasets
validate the effectiveness of our approach, leading to our method
contributing to achieving the Top-1 rank in the DFGC competition