\section{Dataset and Features}
The dataset used for training is derived from the DeepFake Detection Challenge
(DFDC) dataset on Kaggle. The original DFDC dataset comprises over 470GB of mp4
videos. An analysis conducted on a 20GB sample of the dataset revealed that
approximately 83% of the examples are deepfakes. This imbalance arises because
each real example has been deepfaked anywhere from 1 to 22 times, with an average
of 5.19 fakes per real image. This diversity in deepfake-generation techniques
necessitates a robust detection approach.
Each video in the dataset has a frame rate of 30fps and is exactly 10 seconds long.
The videos feature individuals of various races and ages, with backgrounds ranging
from bright indoor settings to dark outdoor scenes. To simplify the problem to image
classification, the dataset was transformed into a collection of uniformly-sized
images, each labeled as REAL or FAKE, with an approximate 80-20 split between real
images and deepfakes.
For this purpose, 5 frames were sampled from each video (at a frequency of 2
seconds or every 60 frames) from a 100GB subset of the original video dataset. Each
image frame was resized to (224x224) pixels, normalized by dividing by 255, and
randomly transformed (in brightness, contrast, and saturation). Additionally, 3-4
deepfakes corresponding to each real image were included in the dataset.