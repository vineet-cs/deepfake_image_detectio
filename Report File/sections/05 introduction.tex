\section{Introduction}
\vspace{1.5cm}
The rapid advancement of deep generative models has led to
significant progress in face forgery techniques. Among these,
DeepFake stands out for its highly realistic synthesis and ease of
deployment. This technique swaps the face of a source identity in an
authentic video with a synthesized face of a target identity, while
maintaining consistent facial attributes such as expression and head
pose. However, the misuse of DeepFake can enable attackers to
fabricate human activities that never occurred, posing serious
threats to societal security and trustworthiness. Therefore,
developing effective DeepFake detection methods is crucial.
\vspace{1.0cm}

Various methods have been proposed to detect DeepFake videos,
relying on clues such as hand-crafted features, semantic cues, or
data-driven approaches. However, DeepFake detection remains
challenging due to two main reasons. Firstly, the constant
improvement of counterfeiting techniques leads to a less subtle
distinction between real and fake videos. Secondly, there is a
significant drop in performance when applying detection methods